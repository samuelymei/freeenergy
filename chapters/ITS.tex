% !TeX spellcheck = en_US
% !TeX encoding = UTF-8
\section{Integrated Tempering Sampling\label{Sec:ES:ITS}}
Integrated tempering sampling (ITS) was proposed by Gao in 2008.\cite{GaoJCP2008}  It starts by writing a generalized distribution function $P(U)$, which is defined as the integration over $\beta$ in temperature space
\begin{equation}
	P(U)=\int f(\beta^\prime) e^{-\beta^\prime U}\diff \beta^\prime,
	\label{Eq:ES:ITS:distribution}
\end{equation}
where $f(\beta^\prime)$ is the weight, and if one takes $f(\beta^\prime)=\Omega(U)\delta(\beta - \beta^\prime)$ with $\Omega(U)$ being the density of states, the above equation returns to the unnormalized normal Boltzmann distribution function
\begin{equation}
	P(U)=\Omega(U)e^{-\beta U}.
\end{equation}
An energy function $U^\prime$ can be defined from this distribution function as
\begin{equation}
	U^\prime=-\beta^{-1}\ln \int f(\beta^\prime) e^{-\beta^\prime U}\diff \beta^\prime
\end{equation}
Correspondingly, the force can be calculated as
\begin{equation}
	\mathbf{F}^\prime=-\frac{\partial U^\prime}{\partial \mathbf{r}}=-\frac{\int \beta^{\prime} f\left(\beta^{\prime}\right) e^{-\beta^{\prime} U} d \beta^{\prime}}{\beta \int f\left(\beta^{\prime}\right) e^{-\beta^{\prime} U} d \beta^{\prime}} \frac{\partial U}{\partial \mathbf{r}}=\frac{\int \beta^{\prime} f\left(\beta^{\prime}\right) e^{-\beta^{\prime} U} d \beta^{\prime}}{\beta \int f\left(\beta^{\prime}\right) e^{-\beta^{\prime} U} d \beta^{\prime}} \mathbf{F} ,
\end{equation}
in which $\mathbf{F}$ is the force calculated from the original potential energy function of the system under study. A molecular dynamics simulation can be conducted using this force at the desired temperature $\beta$. The difficult part of this method is the determination of $f(\beta^\prime)$ that can yield an efficient sampling in the desired energy range.

Let $f(\beta^\prime)$ be a sum of delta functions
\begin{equation}
	f\left(\beta^{\prime}\right)=\sum_{k=1}^N n_k \delta\left(\beta^{\prime}-\beta_k\right) .
\end{equation}
With this, the distribution function in Eq.~\ref{Eq:ES:ITS:distribution} becomes
\begin{equation}
	P(U)=\sum_{k=1}^N n_k  e^{-\beta_k U}
\end{equation}
with weighting factors $n_k$ to be determined. By integration in  the entire configuration space, we find
\begin{equation}
	P_k= n_k \int e^{-\beta_k  U}\diff \mathbf{x}=\sum_{k=1}^N n_k Q_k,
\end{equation}
where $Q_k$ is the partition function at temperature $\beta_k$. One can preselect the ratios between $P_k$'s for all $k$'s between $1$ and $N$, and thus define a set of fixed expectation values $\{p_k^0\}$ for the normalized quantities $p_k$
\begin{equation}
	p_k=P_k/\sum_{k=1}^N P_k.
\end{equation}

However, the partition functions $Q_k$'s are unknown. Therefore, the desired distribution in terms of $\{p_k\}$ in an MD simulation can be achieved in an iterative procedure.
\begin{enumerate}
	\item Select a sequence of $\left\{\beta_k, k=1, N\right\}$ that cover the desired range of temperature.
	\item With a set of initial guess, $n_k(0)$, for $\left\{n_k, k=1, N\right\}$, a molecular dynamics simulations using forces 
	$$
	F_b=\frac{\sum_k \beta^{\prime} n_k e^{-\beta_k U}}{\beta \sum_k n_k e^{-\beta_k U}} F .
	$$
	is conducted
	\item $P_k$'s and thus $p_k$'s are calculated from the trajectory, and a new set of $\left\{n_k(1), k=1, N\right\}$ are obtained from $n_k(1)=n_k(0) p_k^0 / p_k$, followed by normalization.
	\item Return to step (2) and repeat steps (2) and (3) to update $\left\{n_k, k=1, N\right\}$, until $\left\{p_k, k=1, N\right\}$ converges to $\left\{p_k^0, k=1, N\right\}$.
	\item Run MD simulations using the biased potential $U^\prime$ with the final set of $\left\{n_k, k=1, N\right\}$. The thermodynamic properties of the system are then calculated by reweighting of each term by multiplying a factor $e^{\beta\left(U^{\prime}(r)-U(r)\right)}=e^{-\beta U(r)} / P(U)$, with $P(U)$ being defined by Eq. ~\ref{Eq:ES:ITS:distribution}.
\end{enumerate}

Introducing the idea of solute tempering into ITS leads to the selective integrated tempering sampling (SITS).\cite{YangJCP2009}