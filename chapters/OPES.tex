% !TeX spellcheck = en_US
% !TeX encoding = UTF-8
\section{On-the-f{l}y Probability Enhanced Sampling\label{Sec:ES:OPES}}
The On-the-f{l}y Probability Enhanced Sampling (OPES) method was developed by Invernizzi and Parrinello in 2020.\cite{InvernizziJPCL2020} Different from the well-tempered metadynamics (WTMetaD), in which the biasing potential is defined independent of the probability distribution, the core idea of OPES is to update self-consistently the estimate of the probability distributions and of the biasing potential in an on-the-fly fashion. The probability distribution is constructed via reweighting and is used to estimate the biasing potential.

In order to converge to the flat distribution, the biasing potential at the $n^{th}$ step can be defined as
\begin{equation}
	V_n(\mathbf{s})=\frac{1}{\beta} \log{\hat{P}_n(\mathbf{s})},
\end{equation}
where $\hat{P}_n(\mathbf{s})$ is an estimate of the probability. However, in WTMetaD, we aim at reaching a target distribution $P^{tg}(\mathbf{s})$, which can be obtained if the bias is defined as
\begin{equation}
	V(\mathbf{s})=\frac{1}{\beta} \log{\frac{P(\mathbf{s})}{P^{tg}(\mathbf{s})}}.
\end{equation}
In WTMetaD,
\begin{equation}
	P^{tg}(\mathbf{s})\propto \left[P(\mathbf{s})\right]^{1/\gamma},
\end{equation}
which leads to
\begin{equation}
	V(\mathbf{s})=(1-1/\gamma)\frac{1}{\beta} \log{P(\mathbf{s})}.
\end{equation}

During the simulation, the estimated unbiased probability distribution is expressed as weighted sum of the Gaussian hills according to the previously deposited bias potential as
\begin{equation}
	\tilde{P}_n(\mathbf{s})=\frac{\sum_k^n w_k G(\mathbf{s},\mathbf{s}_k)}{\sum_k^n w_k},
\end{equation}
where the weights $w_k=e^{\beta V_{k-1}(\mathbf{s}_k)}$. Different from the (well-tempered) metadynamics, the deposited Gaussian here does not play a role of biasing potential, but servers as the density estimate with a Gaussian kernel. The tilde above $P$ indicates that the probability distribution is not normalized. The normalization should be carried out with respect to the CV space, denoted as $\Omega_n$, actually explored up to step $n$. Thus, the normalization factor is
\begin{equation}
	Z_n=\frac{1}{|\Omega_n|}\int_{\Omega_n}\tilde{P}_n(\mathbf{s})\diff \mathbf{s},
\end{equation}
with $|\Omega_n|$ being the volume explored in the CV space.

With an additional positive term $\epsilon \ll 1$, the bias at the $n${th} step becomes
\begin{equation}
	V(\mathbf{s})=(1-1/\gamma)\frac{1}{\beta} \log{\left(\frac{\tilde{P}_n(\mathbf{s})}{Z_n}+\epsilon\right)}.
\end{equation}
$\epsilon$ can also allow one to set a limit on the bias, thus providing better control over the desired exploration. It can be chosen to be $\epsilon=e^{-\beta \Delta E/(1-1/\gamma)}$, where $\Delta E$ is the height of the free energy barriers one wishes to overcome during the enhanced sampling. It leads to
\begin{equation}
	V(\mathbf{s})= -\Delta E
\end{equation}
without depositing biasing potential.

As an important feature, the reweighting process is simple and straightforward, which can be performed as in the usual umbrella sampling
\begin{equation}
	P(\mathbf{s})=\frac{\left\langle\delta[\mathbf{s}-\mathbf{s}(\mathbf{R})] \mathrm{e}^{\beta V(\mathbf{s})}\right\rangle_V}{\left\langle\mathrm{e}^{\beta V(\mathbf{s})}\right\rangle_V}.
\end{equation}