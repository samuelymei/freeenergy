% !TeX spellcheck = en_US
\section{Multidimensional Scaling\label{Sec:DR:MDS}}
Multidimensional scaling (MDS) is a method that represents measurements of similarity (or dissimilarity) among pairs of objects as distances between points of a low-dimensional multidimensional space. Since MDS is often used for data visualization, the mapped space usually has a very low dimension, for instance 2.

There are two main variations of MDS: metric MDS and non-metric MDS. Metric MDS aims to preserve the actual distances or dissimilarities between objects as accurately as possible in the lower-dimensional space. Metric MDS assumes that the pairwise distances or dissimilarities are metric, meaning they satisfy the triangle inequality. It uses techniques such as eigenvalue decomposition or optimization algorithms to find the configuration of points that minimizes the difference between the original distances and the distances in the lower-dimensional space. Non-metric MDS, also known as ordinal MDS, does not assume that the pairwise distances or dissimilarities are metric. Instead, it focuses on preserving the ordinal relationships between objects, meaning it tries to maintain the rank order of distances or dissimilarities rather than their actual values.