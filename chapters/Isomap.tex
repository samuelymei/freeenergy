% !TeX spellcheck = en_US
\section{Isometric Feature Mapping (Isomap)\label{Sec:DR:Isomap}}
Isomap, introduced in 2000 for the first time, is a nonlinear generalization of the MDS algorithm in which Euclidean distances are replaced by geodesic distances.\cite{TenenhaumScience2000} Isomap seeks a mapping such that the geodesic distance between data points match the corresponding Euclidean distance in the transformed space. However, the geometric structure of the given data is unknown usually. In order to obtain the geodesic distance between the points, it has been assumed that, in a small neighborhood, the Euclidean distance is a good approximation for the geodesic distance. While for the points far apart, the geodesic distance is approximated as the sum of Euclidean distances along the shortest connecting path.

The first step is to build a weighted neighborhood graph $G(\mathcal{V},\mathcal{E})$ from the given data by connecting only nearby points, where the vertices or nodes, $\mathcal{V}=\{\mathbf{x}_1,\dots,\mathbf{x}_n\}$, are the input data and the edges, $\mathcal{E}=\{e_{ij}\}$, indicate the neighborhood relationship between the points. The weight $w_{ij}$ of edge $e_{ij}$ equals to the distance $d_{ij}$ between those points if they are close to each, or $0$ otherwise. Closeness is defined either by the $\epsilon$-approach, if $\lVert\mathbf{x}_i-\mathbf{x}_j\rVert<\epsilon$ where $\epsilon >0$, or by the $K$-nearest neighbors. Then, Dijkstra’s algorithm or Floyd's algorithm is applied with the nearest neighbor graph $G$ to find the shortest-path distances ($d_G(i,j)$) for all pairs of data points. Finally, MDS is applied to the distance matrix $d_G(i,j)$ to find a $k$-dimensional representation $\mathbf{Y}$ of the original data.