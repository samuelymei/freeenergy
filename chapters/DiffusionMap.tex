% !TeX spellcheck = en_US
\section{Diffusion Map\label{Sec:DR:DM}}
A diffusion map was developed by Coifman et al. in 2005\cite{CoifmanPNAS2005a,CoifmanPNAS2005b,CoifmanACHA2006}, which computes a family of embeddings of a data set into Euclidean space (often low-dimensional) whose coordinates can be computed from the eigenvectors and eigenvalues of a diffusion operator on the data, while ensuring that the diffusion distance in the original space between points is well approximated by the Euclidean distance in the reduced-dimensional space. Diffusion maps are part of the family of nonlinear dimensionality reduction methods which focus on discovering the underlying manifold that the data has been sampled from. By integrating local similarities at different scales, diffusion maps give a global description of the data-set. Compared with other methods, the diffusion map algorithm is robust to noise perturbation and computationally inexpensive.

Given a set of $N$ data points $\mathbf{x}=\{x_1, x_2,\dots,x_N\}$, the connectivity between data points $x_i$ and $x_j$ is defined by the transition probability between these two points $p(x_i,x_j)$, which is measured by their distance
\begin{equation}
	p(x_i,x_j)\propto \begin{cases}
		k(x_i,x_j), & \text{if } k(x_i,x_j) >\epsilon\\
		0, & \text{otherwise}
	\end{cases},
\end{equation}
with $0<\epsilon \ll 1$. The kernel $k(x,y)$ defines a local measure of similarity within a certain neighborhood. Since a given kernel will capture a specific feature of the data set, it should be well-designed to match the requirement of the application. A Gaussian kernel
\begin{equation}
	k(x,y)=\exp{\left(-\frac{|x-y|^2}{\alpha}\right)}
\end{equation}
is frequently used, where $\alpha$ tunes the size of the neighborhood. Transition probability matrix $\mathbf{P}$ must be row-normalized, which leads to
\begin{equation}
	p(x_i,x_j)= \begin{cases}
		k(x_i,x_j)/\sum\limits_{x_j\in S_i}k(x_i,x_j), & \text{if } k(x_i,x_j) >\epsilon\\
		0, & \text{otherwise}
	\end{cases}.
\end{equation}
$S_i$ is the neighborhood of $x_i$.
 
Suppose there are three data points $\{x_1, x_2, x_3\}$ and the single-step transition probability matrix is
\begin{equation}
  P=\begin{bmatrix}
	p_{11} & p_{12} & 0\\
	p_{21} & p_{22} & p_{23} \\
	0      & p_{32} & p_{33}
\end{bmatrix},
\end{equation}
where we have assumed that the transition probability from $x_1$ to $x_3$ is zero and vice versa. It can be easily found that
\begin{equation}
  P^2=
  \begin{bmatrix}
  	p_{11}p_{11}+p_{12}p_{21} & p_{11}p_{12}+p_{12}p_{22} & p_{12}p_{23}\\
  	p_{21}p_{11}+p_{22}p_{21} & p_{21}p_{12}+p_{22}p_{22}+p_{23}p_{32} & p_{22}p_{23}+p_{23}p_{33} \\
  	p_{32}p_{21}      & p_{32}p_{22}+p_{33}p_{32} & p_{32}p_{23}+p_{33}p_{33}
  \end{bmatrix}.
\end{equation}
It shows that after two steps of propagation, the probability of transition from $x_1$ to $x_3$ is no longer zero, instead it equals to the probability of transition from $x_1$ to $x_2$, $p_{12}$, in the first step multiplied by the probability of transition from $x_2$ to $x_3$, $p_{23}$, in the second step. Similar observation can also be found for other transitions among the data points.

It can be seen from the above that $P^t_{ij}$ sum all paths of length $t$ from point $x_i$ to point $x_j$. With increasing value of $t$, different scales of the data structure can be visualized. This is the diffusion process, during which the local connectivity is integrated to present the global connectivity of a data set. Besides, pathways built by long, low probability jumps are gradually replaced by short, high probability jumps. 

With the global geometric structure of a data set uncovered by the diffusion process described above, a diffusion metric can be defined for this structure. The metric measures the similarity of two points in the observed space as the connectivity between them, and is defined as
\begin{align}
	D_t(x_i,x_j)^2=&\sum_{u\in X}|p_t(x_i,u)-p_t(x_j,u)|^2\notag\\
	              =&\sum_k |P_{ik}^t-P_{kj}^t|^2.
\end{align}
In order to have a small diffusion distance between two points, there should be many high probability paths of length $t$ linking these two points. Since it sums over all possible paths, this algorithm is robust to noise perturbation. Besides, the path probabilities between $x_i$, $u$ and $u$, $x_j$ must be roughly equal. This happens when both $x_i$ and $x_j$ are well connected via $u$. However, calculating diffusion distances is computational expensive. Therefore, it is convenient to map data points into a Euclidean space, in which the diffusion distance becomes the Euclidean distance in this new diffusion space. 

A diffusion map maps coordinates between data and diffusion space by reorganizing data according to the diffusion metric. It preserves a data set's intrinsic geometry with data points mapped to a lower-dimensional structure. With the mapping
\begin{equation}
	y_i := \begin{bmatrix}
		p_t(x_i, x_{1}) \\
		p_t(x_i, x_{2}) \\
		\vdots \\
		p_t(x_i, x_{1N})
	\end{bmatrix}=P_{i\ast}^{\operatorname{T}},
\end{equation}
the Euclidean distance between two mapped points, $y_i$ and $y_j$, is
\begin{align}
	\lVert y_i-y_j\rVert_E^2=&\sum_{u\in X} |p_t(x_i,u)-p_t(x_j,u)|^2\notag\\
	                        =&\sum_k |P_{ik}^t-P_{kj}^t|^2\notag\\
	                        =&D_t(x_i,x_j)^2.
\end{align}

To reduce the dimensionality, we define the normalized diffusion matrix
\begin{equation}
	P=D^{-1}K,
\end{equation}
where $D$ is the diagonal matrix consisting of the row-sums of $K$. With the eigenvectors $\phi_i$ and $\lambda_i$ of $P$, the mapped points can be rewritten as
\begin{equation}
	y_i^\prime=\begin{bmatrix}
		\lambda_1^t\phi_1(i) \\
		\lambda_2^t\phi_2(i) \\
		\vdots \\
		\lambda_n^t\phi_n(i)
	\end{bmatrix},
\end{equation}
where $\phi_k(s)$ indicates the $s$th element of the $k$th eigenvector of $P$. The Euclidean distance between mapped points $y_i^\prime$ and $y_j^\prime$ is the diffusion distance. Dimension reduction is achieved by retaining the $m$ dimensions associated with the dominant eigenvectors.