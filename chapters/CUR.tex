% !TeX spellcheck = en_US
\section{CUR Decomposition\label{Sec:DR:CUR}}
CUR decomposition, developed by Mahoney and Drineas\cite{MahoneyPNAS2009}, finds a low-rank approximation of matrix $A$ as the product of three matrices $C$, $U$, and $R$, where $C$ is a matrix consisting of selected columns of the original matrix, $R$ is a matrix consisting of selected rows of the original matrix, and $U$ is a matrix that ideally reconstructs the original matrix from $C$ and $R$. Usually the CUR is designed to be a rank-$k$ approximation, which requires that $C$ contains $k$ columns of $A$, $R$ contains $k$ rows of $A$, and $U$ is a $k$-by-$k$ matrix. The CUR matrix decomposition technique was developed to provide more interpretable and computationally efficient alternatives to SVD in principal component analysis (PCA), despite the fact that CUR is usually less accurate than SVD.

The fundamental questions of the CUR decomposition methods are: 1) Which columns of $A$ should be used to build $C$? Which rows should be used for $R$? 2) How to obtain the best $U$ given $C$ and $R$? %Different strategies have been proposed, for instance column pivoted QR factorizations, volume optimization, uniform sampling of columns, leverage scores, and empirical interpolation approaches. 