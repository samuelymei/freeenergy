% !TeX spellcheck = en_US
\section{Linear Discriminant Analysis\label{Sec:DR:LDA}}
Linear discriminant analysis (LDA), aka Fisher linear discriminant analysis, was originally developed in 1936 by Ronald A. Fisher\cite{FisherAE1936}. It is a dimensionality reduction method for classification problems that preserves as much of the class discriminatory information as possible by maximizing the ratio of the between-class variance to the within-class variance. Closely related to PCA, LDA is also based on linear transformations.

Given the original data set $\mathbf{X}=\left\{\mathbf{x}_1, \mathbf{x}_2,\cdots,\mathbf{x}_N\right\}$, where $\mathbf{x}_i$ represents the $i^{th}$ sample with $M$ features ($\mathbf{x}_i\in \mathbb{R}^M$), and $N$ is the total number of samples. Assume the data samples are catagorized into $C$ classes, $\mathbf{X}=\left[\boldsymbol{\omega_1}, \boldsymbol{\omega_2}, \cdots, \boldsymbol{\omega_C}\right]$, and class $j$ contains $n_j$ samples. The sum of $n_j$ equals to the  total number of samples.
\begin{equation}
	N=\sum_{j=1}^C n_j.
\end{equation}
LDA seeks to obtain a transformation of $\mathbf{X}$ to $\mathbf{Y}$ through projecting the samples in $\mathbf{X}$ onto a hyperplane with dimension $C-1$. The sample mean for class $\boldsymbol{\mu_j}$ is calculated as
\begin{equation}
    \boldsymbol{\mu_j}=\frac{1}{n_j}\sum_{\mathbf{x}_i\in \boldsymbol{\omega_j}} \mathbf{x}_i,
\end{equation} 
and the sample mean of all the classes is computed as
\begin{equation}
    \boldsymbol{\mu}=\frac{1}{N}\sum_{i=1}^{N} \mathbf{x}_i=\sum_{j=1}^C \frac{n_j}{N}\boldsymbol{\mu}_j.
\end{equation}
Their projections, $\mathbf{m}_i$ and $\mathbf{m}$, are computed via
\begin{equation}
    \mathbf{m}_j=\mathbf{W}^{\operatorname{T}}\boldsymbol{\mu}_j
\end{equation}
and
\begin{equation}
	\mathbf{m}=\mathbf{W}^{\operatorname{T}}\boldsymbol{\mu},
\end{equation}
where $\mathbf{W}$ is the transformation matrix of LDA.

To calculate the between-class variance (scatter) $\mathbf{S}_B$, the separation distance between different classes will be calculated by
\begin{equation}
    \mathbf{S}_B=\sum_{j=1}^C n_j\mathbf{S}_{Bj},
\end{equation}
where
\begin{equation}
	\mathbf{S}_{Bj}=(\boldsymbol{\mu}_j-\boldsymbol{\mu})(\boldsymbol{\mu}_j-\boldsymbol{\mu})^{\operatorname{T}}.
\end{equation}
In the projected space,
\begin{equation}
	(\mathbf{m}_j-\mathbf{m})(\mathbf{m}_j-\mathbf{m})^{\operatorname{T}}=\mathbf{W}^{\operatorname{T}}\mathbf{S}_{Bj}\mathbf{W}.
\end{equation}

The within-class variance represents the difference between the mean and the samples within each class. The within-class variance (scatter) of each class $\mathbf{S}_{Wj}$ is calculated as
\begin{align}
	\sum_{j=1}^C\sum_{\mathbf{x}_i\in \boldsymbol{\omega_j}}&\left(\mathbf{x}_i-\mathbf{m}_j\right)\left(\mathbf{x}_i-\mathbf{m}_j\right)^{\operatorname{T}}\notag\\
	=&\sum_{j=1}^C\sum_{\mathbf{x}_i\in \boldsymbol{\omega_j}} \left(\mathbf{W}^{\operatorname{T}}\mathbf{x}_i-\mathbf{W}^{\operatorname{T}}\boldsymbol{\mu}_j\right)\left(\mathbf{W}^{\operatorname{T}}\mathbf{x}_i-\mathbf{W}^{\operatorname{T}}\boldsymbol{\mu}_j\right)^{\operatorname{T}}\notag\\
	=&\sum_{j=1}^C\sum_{\mathbf{x}_i\in \boldsymbol{\omega_j}} \mathbf{W}^{\operatorname{T}}\left(\mathbf{x}_i-\boldsymbol{\mu}_j\right)\left(\mathbf{x}_i-\boldsymbol{\mu}_j\right)^{\operatorname{T}}\mathbf{W}\notag\\
	=&\sum_{j=1}^C\mathbf{W}^{\operatorname{T}}\mathbf{S}_{Wj}\mathbf{W}\notag\\
	=&\mathbf{W}^{\operatorname{T}}\mathbf{S}_{W}\mathbf{W}
\end{align}

The transformation matrix $\mathbf{W}$ can be calculated by maximizing the ratio of the determinant of $\mathbf{S}_B$ to the determinant of $\mathbf{S}_W$ in the projected space (known as Fishers criterion) 
\begin{equation}
	\operatorname*{arg\,max}\limits_{\substack{\mathbf{W}}}\frac{|\mathbf{W}^{\operatorname{T}}\mathbf{S}_B\mathbf{W}|}{|\mathbf{W}^{\operatorname{T}}\mathbf{S}_W\mathbf{W}|}.
\end{equation}
Note that the determinant of the co-variance matrix tells us how much variance a class has, and it has the same value under any ortho-normal projection. So Fishers criterion tries to find the projection that maximizes the variance of the class means and minimzses the variance of the individual classes.

This optimization problem can have infinitely number of solutions with the same objective function value, due that for a solution $\mathbf{W}$ all the vectors $c\cdot\mathbf{W}$ give exactly the same value for the objective function. Without loss of generality, the constraint
\begin{equation}
	\mathbf{W}^{\operatorname{T}}\mathbf{S}_W\mathbf{W}=1
\end{equation}
can be applied. Then the problem becomes
\begin{equation}
	\operatorname*{arg\,max}\limits_{\substack{\mathbf{W}}}\mathbf{W}^{\operatorname{T}}\mathbf{S}_B\mathbf{W}
\end{equation}
$\text{s.t.}$
\begin{equation}
	\mathbf{W}^{\operatorname{T}}\mathbf{S}_W\mathbf{W}=1.
\end{equation}

The Lagrangian for this optimization is
\begin{equation}
	\mathcal{L}_{LDA}=\mathbf{W}^{\operatorname{T}}\mathbf{S}_B\mathbf{W}-\lambda(\mathbf{W}^{\operatorname{T}}\mathbf{S}_W\mathbf{W}-1).
\end{equation}
Minimizing  $\mathcal{L}_{LDA}$ with respect to $\mathbf{W}$ leads to
\begin{equation}
	\mathbf{S}_B \mathbf{W}=\mathbf{S}_W \mathbf{W}\Lambda.
\end{equation}
Multiplying on both sides by the inverse of $\mathbf{S}_W$ (if $\mathbf{S}_W$ is a non-singular), it becomes
\begin{align}
	{\mathbf{S}_W}^{-1}\mathbf{S_B} \mathbf{W}=\mathbf{W}\Lambda.
\end{align}
Then the Fisher’s criterion is maximized when the projection matrix $\mathbf{W}$ is composed of the eigenvectors of ${\mathbf{S}_W}^{-1}\mathbf{S_B}$, and $\Lambda$ are the associated eigenvalues. Notice that there will be at most $C-1$ eigenvectors with non-zero real corresponding eigenvalues. Each eigenvectors represents one axis of the LDA space, and the corresponding eigenvalues represents the discriminatory ability between different classes. Thus, the eigenvectors with the $k$ highest eigenvalues are used to construct a lower dimensional space.