\chapter{Dimension Reduction\label{chapter:DR}}
\begin{chapquote}{Winston Churchill%, \textit{\url{https://en.wikiquote.org/wiki/Albert_Einstein}}
	}
	``Success is stumbling from failure to failure with no loss of enthusiasm.''
\end{chapquote}

Natural data is often represented in high-dimensional spaces, leading to the ``curse of dimensionality'' challenge in analysis. This complexity makes it difficult for humans to visualize and interpret such data. However, The \textbf{manifold hypothesis} posits that many high-dimensional data sets that occur in the real world actually lie along low-dimensional latent manifolds inside that high-dimensional space. Identifying slowly varying order parameters, or collective variables (CVs), is crucial in the field of physical chemistry, especially for complex systems. These CVs are typically expected to exist on a low-dimensional manifold, capturing the slow dynamics of rare events amid a range of faster occurrences. Identifying effective CVs is challenging, often relying on intuition. According to Peters\cite{PetersARPC2016}, an ideal CV should meet three criteria: (i) it should be a function of the instantaneous configuration space, excluding velocities; (ii) its value should change monotonically between two states, with corresponding isosurfaces creating non-intersecting dividing surfaces in the configuration space; (iii) it allows for projecting a free energy profile along it, ensuring that the reduced dynamics along the CV are consistent with those in the full phase space.

For a brief introduction to the modern dimension reduction methods or manifold learning methods, please refer to Ref.~\cite{IzamanWIREsCS2012}. Specially, the generic problem addressed by dimension reduction, or manifold learning, is as follows: given a set of $k$ observations $\mathbf{x}_1,\dots,\mathbf{x}_k$ in $\mathbb{R}^D$, find a set of points $\mathbf{y}_1,\dots,\mathbf{y}_k$ in $\mathbb{R}^d$ ($d\ll D$) that serve as the best representation of $\mathbf{x}_i$. For dynamic systems, the core of manifold learning methods for dimension reduction is to construct a random walk through Markov chain on the data set, where the transition probabilities of $p_{kl}$ depend on a kernal function and the distances between samples. Most dimensionality reduction algorithms fit into either one of two broad categories: Matrix factorization (such as PCA) or Graph layout (such as t-SNE and UMAP).

But, how can one gauge a dimensionality reduction algorithm's performance effectively? The answer may not be unique, and some metrics should be considered.
\begin{itemize}
	\item \textbf{Data reconstruction error} is the difference between the original data and the reconstructed data, which is obtained by applying the inverse transformation of the dimensionality reduction algorithm. The lower the data reconstruction error, the better the algorithm is at retaining the essential features of the original data. The data reconstruction error can be quantified in different ways, such as mean squared error, root mean squared error, or mean absolute error.
	
	\item \textbf{Data compression ratio} is the ratio of the size of the original data to the size of the reduced data. The higher the data compression ratio, the more efficient the algorithm is at reducing the dimensionality of the data. However, large data compression ratio often leads to high data reconstructure error. Therefore, one should always balance the trade-off between data compression and data reconstruction when choosing a dimensionality reduction algorithm.
	
	\item \textbf{Data visualization quality} can be used when the dimensionality of the data is reduced to two or three dimensions, which can be easily plotted and visualized with bare eyes to find how well the reduced data captures the patterns, clusters, and outliers in the original data.
	
	\item \textbf{Data classification accuracy} is the proportion of correctly predicted labels out of the total number of labels for a supervised learning problem. Different classifiers, such as logistic regression, k-nearest neighbors, or support vector machines, can be used to test the data classification accuracy. The higher the data classification accuracy, the better the algorithm is at preserving the discriminative power of the data.
	
	\item \textbf{Algorithm complexity and scalability}, such as time complexity, space complexity, or iteration complexity, refers to how fast and how well the algorithm can handle large and high-dimensional datasets. The algorithm complexity and scalability depend on factors such as the computational cost, the memory usage, and the convergence rate of the algorithm. 
	
	\item \textbf{Algorithm suitability and robustness} means how well the algorithm fits the characteristics and the objectives of the data and the problem. The algorithm suitability and robustness depend on various aspects such as the data type, the data distribution, the data noise, and the data interpretation. One can use various methods to test the algorithm suitability and robustness, such as cross-validation, sensitivity analysis, or parameter tuning.
	
\end{itemize}

\clearpage 
% !TeX spellcheck = en_US
\section{Principal Component Analysis\label{Sec:DR:PCA}}
Principal component analysis (PCA) was first developed by Hotelling in 1933,\cite{HotellingJEP1933} which is one of the dimension reduction methods that \textbf{linearly} transforms a data set consisting of a large number of interrelated variables to a new set of uncorrelated variables, the principal components (PCs), while retaining as much as possible of the variation present in the data set. The output PCs are ordered so that the first few retain most of the variation present in all of the original variables. There have been many excellent review and tutorial of this method. For a (probably) most recent one, please refer to Ref.~\cite{GreenacreNRMP2022}

%Principal component analysis is a versatile statistical method that combines information linearly from $n$ variables observed on the same objects into $r$ variables, called principal components, where $r\ll n$.

Suppose that $\mathbf{x}$ is a vector of $p$ random variables, of which the covariance matrix is $\boldsymbol{\Sigma}$. When $\boldsymbol{\Sigma}$ is unknown, it is often replaced by a sample variance $\mathbf{S}$. Let $\boldsymbol{\alpha}_k$ (for $k=1,\dots, p$ ) be the $k$th eigenvector of $\boldsymbol{\Sigma}$ corresponding to its $k$th largest eigenvalue of $\lambda_k$. The coordinate on the $k$th PC can be written as
\begin{equation}
	z_k={\boldsymbol{\alpha}_k}^{\operatorname{T}} \mathbf{x}=\sum_{j=1}^p \alpha_{kj}x_j,
\end{equation}
where $\operatorname{T}$ denotes transpose. Normally, $\boldsymbol{\alpha}_k$ is chosen to have a unit length (\textit{i.e.} ${\boldsymbol{\alpha}_k}^{\operatorname{T}}\boldsymbol{\alpha}_k=1$). Then the variance of $z_k$, $\operatorname{var}(z_k)$, equals to $\lambda_k$.

To derive the form of the PCs, first consider $\boldsymbol{\alpha}_1$, which maximizes $\operatorname{var}\left[{\boldsymbol{\alpha}_1}^{\operatorname{T}}\mathbf{x}\right]={\boldsymbol{\alpha}_1}^{\operatorname{T}}\boldsymbol{\Sigma}\boldsymbol{\alpha}_1$ subject to ${\boldsymbol{\alpha}_1}^{\operatorname{T}}\boldsymbol{\alpha}_1=1$. Using the technique of Lagrange multipliers, it becomes to maximize
\begin{equation}
	{\boldsymbol{\alpha}_1}^{\operatorname{T}}\boldsymbol{\Sigma}\boldsymbol{\alpha}_1-\lambda ({\boldsymbol{\alpha}_1}^{\operatorname{T}}\boldsymbol{\alpha}_1-1),
\end{equation}
where $\lambda$ is a Lagrange multiplier. Differentiation with respect to $\boldsymbol{\alpha}_1$ gives
\begin{equation}
	\boldsymbol{\Sigma}\boldsymbol{\alpha}_1-\lambda \boldsymbol{\alpha}_1=0.
\end{equation}
Thus, $\lambda$ is an eigenvalue of $\boldsymbol{\Sigma}$, and $\boldsymbol{\alpha}_1$ is the corresponding eigenvector. Also note that
\begin{equation}
	{\boldsymbol{\alpha}_1}^{\operatorname{T}}\boldsymbol{\Sigma} \boldsymbol{\alpha}_1={\boldsymbol{\alpha}_1}^{\operatorname{T}} \lambda \boldsymbol{\alpha}_1=\lambda {\boldsymbol{\alpha}_1}^{\operatorname{T}} \boldsymbol{\alpha}_1=\lambda.
\end{equation}
Therefore, in order to maximize ${\boldsymbol{\alpha}_1}^{\operatorname{T}}\boldsymbol{\Sigma} \boldsymbol{\alpha}_1$, $\lambda$ must be the largest eigenvalue of $\boldsymbol{\Sigma}$.

Now, let us look at the second PC, $\boldsymbol{\alpha}_2\mathbf{x}$, which maximizes ${\boldsymbol{\alpha}_2}^{\operatorname{T}}\boldsymbol{\Sigma} \boldsymbol{\alpha}_2$ subject to being uncorrelated with the first PC, $\boldsymbol{\alpha}_1\mathbf{x}$, \textit{i.e.} $\operatorname{cov}\left[{\boldsymbol{\alpha}_1}^{\operatorname{T}} \mathrm{x}, {\boldsymbol{\alpha}_2}^{\operatorname{T}} \mathrm{x}\right]=0$. Since
\begin{equation}
	\operatorname{cov}\left[{\boldsymbol{\alpha}_1}^{\operatorname{T}} \mathrm{x}, {\boldsymbol{\alpha}_2}^{\operatorname{T}} \mathrm{x}\right]={\boldsymbol{\alpha}_1}^{\operatorname{T}} \boldsymbol{\Sigma} \boldsymbol{\alpha}_2={\boldsymbol{\alpha}_2}^{\operatorname{T}} \boldsymbol{\Sigma} \boldsymbol{\alpha}_1=\lambda_1 {\boldsymbol{\alpha}_2}^{\operatorname{T}} \boldsymbol{\alpha}_1,
\end{equation}
any one of the following equations
\begin{equation}
\begin{array}{rr}
	{\boldsymbol{\alpha}_1}^{\operatorname{T}} \boldsymbol{\Sigma} \boldsymbol{\alpha}_2=0, & {\boldsymbol{\alpha}_2}^{\operatorname{T}} \boldsymbol{\Sigma} \boldsymbol{\alpha}_1=0 \\
	{\boldsymbol{\alpha}_1}^{\operatorname{T}} \boldsymbol{\alpha}_2=0, & {\boldsymbol{\alpha}_2}^{\operatorname{T}} \boldsymbol{\alpha}_1=0
\end{array}
\end{equation}
could be used to specify the constraint. Using, for instance, the last one, as well as the normalization condition, the quantity to be maximized is
\begin{equation}
	{\boldsymbol{\alpha}_2}^{\operatorname{T}} \boldsymbol{\Sigma} \boldsymbol{\alpha}_2-\lambda\left({\boldsymbol{\alpha}_2}^{\operatorname{T}} \boldsymbol{\alpha}_2-1\right)-\phi {\boldsymbol{\alpha}_2}^{\operatorname{T}} \boldsymbol{\alpha}_1,
\end{equation}
where $\lambda$ and $\phi$ are Lagrange multipliers. Differentiation with respect to $\boldsymbol{\alpha}_2$ gives
\begin{equation}
	\boldsymbol{\Sigma} \boldsymbol{\alpha}_2-\lambda \boldsymbol{\alpha}_2-\phi \boldsymbol{\alpha}_1=\mathbf{0}.
\end{equation}
Multiplying on the left by ${\boldsymbol{\alpha}_1}^{\operatorname{T}}$ gives
\begin{equation}
	{\boldsymbol{\alpha}_1}^{\operatorname{T}} \boldsymbol{\Sigma} \boldsymbol{\alpha}_2-\lambda {\boldsymbol{\alpha}_1}^{\operatorname{T}} \boldsymbol{\alpha}_2-\phi {\boldsymbol{\alpha}_1}^{\operatorname{T}} \boldsymbol{\alpha}_1=0.
\end{equation}
Since the first two terms are zero and ${\boldsymbol{\alpha}_1}^{\operatorname{T}} \boldsymbol{\alpha}_1=1$, it leads to $\phi=0$. Therefore,
\begin{equation}
	\boldsymbol{\Sigma} \boldsymbol{\alpha}_2-\lambda \boldsymbol{\alpha}_2=\mathbf{0},
\end{equation}
indicating that $\lambda$ is an eigenvalue of $\boldsymbol{\Sigma}$ again, and $\boldsymbol{\alpha}_2$ the corresponding eigenvector. We can keep on doing this analysis for the third, fourth, $\dots$, $p$th PCs, and show that $\lambda_3$, $\lambda_4$, $\dots$, $\lambda_p$ are the third, fourth largest, $\dots$, and the smallest eigenvalue of $\boldsymbol{\Sigma}$, and $\boldsymbol{\alpha}_3$, $\boldsymbol{\alpha}_4$, $\dots$, $\boldsymbol{\alpha}_p$ are the corresponding eigenvectors. Furthermore,
\begin{equation}
	\operatorname{var}\left[{\boldsymbol{\alpha}_k}^{\operatorname{T}} \mathbf{x}\right]=\lambda_k \quad \text { for } k=1,2, \ldots, p.
\end{equation}
\clearpage
% !TeX spellcheck = en_US
\section{Multidimensional Scaling\label{Sec:DR:MDS}}
Multidimensional scaling (MDS) is a method that represents measurements of similarity (or dissimilarity) among pairs of objects as distances between points of a low-dimensional multidimensional space. Since MDS is often used for data visualization, the mapped space usually has a very low dimension, for instance 2.

There are two main variations of MDS: metric MDS and non-metric MDS. Metric MDS aims to preserve the actual distances or dissimilarities between objects as accurately as possible in the lower-dimensional space. Metric MDS assumes that the pairwise distances or dissimilarities are metric, meaning they satisfy the triangle inequality. It uses techniques such as eigenvalue decomposition or optimization algorithms to find the configuration of points that minimizes the difference between the original distances and the distances in the lower-dimensional space. Non-metric MDS, also known as ordinal MDS, does not assume that the pairwise distances or dissimilarities are metric. Instead, it focuses on preserving the ordinal relationships between objects, meaning it tries to maintain the rank order of distances or dissimilarities rather than their actual values.

Here, we only look at metric MDS. For two vectors $\mathbf{a}$ and $\mathbf{b}$, the distance between them can be written as
\begin{equation}
	d^2(\mathbf{a},\mathbf{b})=(\mathbf{a}-\mathbf{b})^{\operatorname{T}}(\mathbf{a}-\mathbf{b})=\mathbf{a}^{\operatorname{T}}\mathbf{a}+\mathbf{b}^{\operatorname{T}}\mathbf{b}-2\times(\mathbf{a}^{\operatorname{T}}\mathbf{b}),
\end{equation}
where $\mathbf{a}^{\operatorname{T}}\mathbf{b}$ is the scalar product between $\mathbf{a}$ and $\mathbf{b}$. For $n$ observations in a $m$-dimensional space ($\mathbf{x}_1, \mathbf{x}_2, \dots, \mathbf{x}_n\in \mathbb{R}^m$) stored in a matrix denoted by $\mathbf{X}$, which has a shape of $m\times n$, the cross product matrix $\mathbf{S}$ is then obtained as
\begin{equation}
	\mathbf{S}=\mathbf{X}^{\operatorname{T}}\mathbf{X},
\end{equation}
which has a shape of $n\times n$. The squared Euclidean distance matrix
\begin{equation}
	\mathbf{D}=\mathbf{s}\cdot\mathbf{1}^{\operatorname{T}}+\mathbf{1}\cdot\mathbf{s}^{\operatorname{T}}-2\mathbf{S},
	\label{eq:DR:MDS:D_vs_S}
\end{equation}
where $\mathbf{s}=\operatorname{diag}(S_{ii})$ is the $n\times 1$ vector of the diagonal element of $\mathbf{S}$ and $\mathbf{1}$ is a $n\times 1$ vector of 1s. It shows that the cross product matrix $\mathbf{S}$ can be computed from the distance matrix $\mathbf{D}$, which is the basic idea of metric MDS. Clearly, the solutions are not unique, since distances are invariant with respect to any change of origin. Therefore, constraints must be imposed on the calculations of $\mathbf{X}$. An obvious choice is to choose the origin of the distance as the center of gravity of the dimensions.

Defining a $n\times n$ centering matrix
\begin{equation}
	\mathbf{H}=\mathbf{I}_n- \frac{1}{n}\mathbf{1}\cdot\mathbf{1}^{\operatorname{T}},
\end{equation}
the cross-product matrix is obtained from matrix $\mathbf{D}$ as
\begin{equation}
	\tilde{\mathbf{S}}=-\frac{1}{2}\mathbf{HD}\mathbf{H}^{\operatorname{T}}.
	\label{eq:DR:MDS:tildeS}
\end{equation}
To show the relationship between $\tilde{\mathbf{S}}$ and $\mathbf{S}$, let us take Eq.~\ref{eq:DR:MDS:D_vs_S} into Eq.~\ref{eq:DR:MDS:tildeS}, we find
\begin{equation}
	\tilde{\mathbf{S}}=-\frac{1}{2}\mathbf{H}\left(\mathbf{s}\cdot\mathbf{1}^{\operatorname{T}}+\mathbf{1}\cdot\mathbf{s}^{\operatorname{T}}-2\mathbf{S}\right)\mathbf{H}^{\operatorname{T}}
\end{equation}
Note that $\mathbf{1}^{\operatorname{T}}\cdot \mathbf{H}^{\operatorname{T}}=\mathbf{1}^{\operatorname{T}}\cdot\left(\mathbf{I}_n- \frac{1}{n}\mathbf{1}\cdot\mathbf{1}^{\operatorname{T}}\right)=0$, it yields
\begin{equation}
	\tilde{\mathbf{S}}=\mathbf{HS}\mathbf{H}^{\operatorname{T}}.
\end{equation}
The eigen-decomposition of this matrix gives
\begin{equation}
	\tilde{\mathbf{S}}=\mathbf{U\Lambda}\mathbf{U}^{\operatorname{T}},
\end{equation}
where $\mathbf{U}\mathbf{U}^{\operatorname{T}}=\mathbf{I}$ and $\mathbf{\Lambda}$ is a diagonal matrix of eigenvalues. The best Euclidean approximation of the original distance matrix is thus obtained as
\begin{equation}
	\tilde{\mathbf{Y}}=\mathbf{\Lambda}^{\operatorname{\frac{1}{2}}}\mathbf{U}^{\operatorname{T}}.
\end{equation}
In practice, one often chooses the top $k$ nonzero eigenvalues of $\tilde{\mathbf{S}}$ and build a $k$-dimensional Euclidean embedding of data $\tilde{\mathbf{Y}}_k={\mathbf{\Lambda}_k}^{1/2}{\mathbf{U}_k}^{\operatorname{T}}$, where
\begin{align*}
	{\mathbf{U}_k}^{\operatorname{T}}=&\left[u_1, \dots, u_k\right]^{\operatorname{T}},\quad u_k\in \mathbb{R}^n,\\
	\mathbf{\Lambda}_k=&\operatorname{diag}(\lambda_1,\dots,\lambda_k)
\end{align*}
with $\lambda_1\ge \lambda_2\ge \dots \ge \lambda_k>0$.
\clearpage
% !TeX spellcheck = en_US
\section{Linear Discriminant Analysis\label{Sec:DR:LDA}}
Linear discriminant analysis (LDA), aka Fisher linear discriminant analysis, was originally developed in 1936 by Ronald A. Fisher\cite{FisherAE1936}. It is a dimensionality reduction method for classification problems that preserves as much of the class discriminatory information as possible by maximizing the ratio of the between-class variance to the within-class variance. Closely related to PCA, LDA is also based on linear transformations.

Given the original data set $\mathbf{X}=\left\{\mathbf{x}_1, \mathbf{x}_2,\cdots,\mathbf{x}_N\right\}$, where $\mathbf{x}_i$ represents the $i^{th}$ sample with $M$ features ($\mathbf{x}_i\in \mathbb{R}^M$), and $N$ is the total number of samples. Assume the data samples are catagorized into $C$ classes, $\mathbf{X}=\left[\boldsymbol{\omega_1}, \boldsymbol{\omega_2}, \cdots, \boldsymbol{\omega_C}\right]$, and class $j$ contains $n_j$ samples. The sum of $n_j$ equals to the  total number of samples.
\begin{equation}
	N=\sum_{j=1}^C n_j.
\end{equation}
LDA seeks to obtain a transformation of $\mathbf{X}$ to $\mathbf{Y}$ through projecting the samples in $\mathbf{X}$ onto a hyperplane with dimension $C-1$. The sample mean for class $\boldsymbol{\mu_j}$ is calculated as
\begin{equation}
    \boldsymbol{\mu_j}=\frac{1}{n_j}\sum_{\mathbf{x}_i\in \boldsymbol{\omega_j}} \mathbf{x}_i,
\end{equation} 
and the sample mean of all the classes is computed as
\begin{equation}
    \boldsymbol{\mu}=\frac{1}{N}\sum_{i=1}^{N} \mathbf{x}_i=\sum_{j=1}^C \frac{n_j}{N}\boldsymbol{\mu}_j.
\end{equation}
Their projections, $\mathbf{m}_i$ and $\mathbf{m}$, are computed via
\begin{equation}
    \mathbf{m}_j=\mathbf{W}^{\operatorname{T}}\boldsymbol{\mu}_j
\end{equation}
and
\begin{equation}
	\mathbf{m}=\mathbf{W}^{\operatorname{T}}\boldsymbol{\mu},
\end{equation}
where $\mathbf{W}$ is the transformation matrix of LDA.

To calculate the between-class variance (scatter) $\mathbf{S}_B$, the separation distance between different classes will be calculated by
\begin{equation}
    \mathbf{S}_B=\sum_{j=1}^C n_j\mathbf{S}_{Bj},
\end{equation}
where
\begin{equation}
	\mathbf{S}_{Bj}=(\boldsymbol{\mu}_j-\boldsymbol{\mu})(\boldsymbol{\mu}_j-\boldsymbol{\mu})^{\operatorname{T}}.
\end{equation}
In the projected space,
\begin{equation}
	(\mathbf{m}_j-\mathbf{m})(\mathbf{m}_j-\mathbf{m})^{\operatorname{T}}=\mathbf{W}^{\operatorname{T}}\mathbf{S}_{Bj}\mathbf{W}.
\end{equation}

The within-class variance represents the difference between the mean and the samples within each class. The within-class variance (scatter) of each class $\mathbf{S}_{Wj}$ is calculated as
\begin{align}
	\sum_{j=1}^C\sum_{\mathbf{x}_i\in \boldsymbol{\omega_j}}&\left(\mathbf{x}_i-\mathbf{m}_j\right)\left(\mathbf{x}_i-\mathbf{m}_j\right)^{\operatorname{T}}\notag\\
	=&\sum_{j=1}^C\sum_{\mathbf{x}_i\in \boldsymbol{\omega_j}} \left(\mathbf{W}^{\operatorname{T}}\mathbf{x}_i-\mathbf{W}^{\operatorname{T}}\boldsymbol{\mu}_j\right)\left(\mathbf{W}^{\operatorname{T}}\mathbf{x}_i-\mathbf{W}^{\operatorname{T}}\boldsymbol{\mu}_j\right)^{\operatorname{T}}\notag\\
	=&\sum_{j=1}^C\sum_{\mathbf{x}_i\in \boldsymbol{\omega_j}} \mathbf{W}^{\operatorname{T}}\left(\mathbf{x}_i-\boldsymbol{\mu}_j\right)\left(\mathbf{x}_i-\boldsymbol{\mu}_j\right)^{\operatorname{T}}\mathbf{W}\notag\\
	=&\sum_{j=1}^C\mathbf{W}^{\operatorname{T}}\mathbf{S}_{Wj}\mathbf{W}\notag\\
	=&\mathbf{W}^{\operatorname{T}}\mathbf{S}_{W}\mathbf{W}
\end{align}

The transformation matrix $\mathbf{W}$ can be calculated by maximizing the ratio of the determinant of $\mathbf{S}_B$ to the determinant of $\mathbf{S}_W$ in the projected space (known as Fishers criterion) 
\begin{equation}
	\operatorname*{arg\,max}\limits_{\substack{\mathbf{W}}}\frac{|\mathbf{W}^{\operatorname{T}}\mathbf{S}_B\mathbf{W}|}{|\mathbf{W}^{\operatorname{T}}\mathbf{S}_W\mathbf{W}|}.
\end{equation}
Note that the determinant of the co-variance matrix tells us how much variance a class has, and it has the same value under any ortho-normal projection. So Fishers criterion tries to find the projection that maximizes the variance of the class means and minimzses the variance of the individual classes.

This optimization problem can have infinitely number of solutions with the same objective function value, due that for a solution $\mathbf{W}$ all the vectors $c\cdot\mathbf{W}$ give exactly the same value for the objective function. Without loss of generality, the constraint
\begin{equation}
	\mathbf{W}^{\operatorname{T}}\mathbf{S}_W\mathbf{W}=1
\end{equation}
can be applied. Then the problem becomes
\begin{equation}
	\operatorname*{arg\,max}\limits_{\substack{\mathbf{W}}}\mathbf{W}^{\operatorname{T}}\mathbf{S}_B\mathbf{W}
\end{equation}
$\text{s.t.}$
\begin{equation}
	\mathbf{W}^{\operatorname{T}}\mathbf{S}_W\mathbf{W}=1.
\end{equation}

The Lagrangian for this optimization is
\begin{equation}
	\mathcal{L}_{LDA}=\mathbf{W}^{\operatorname{T}}\mathbf{S}_B\mathbf{W}-\lambda(\mathbf{W}^{\operatorname{T}}\mathbf{S}_W\mathbf{W}-1).
\end{equation}
Minimizing  $\mathcal{L}_{LDA}$ with respect to $\mathbf{W}$ leads to
\begin{equation}
	\mathbf{S}_B \mathbf{W}=\mathbf{S}_W \mathbf{W}\Lambda.
\end{equation}
Multiplying on both sides by the inverse of $\mathbf{S}_W$ (if $\mathbf{S}_W$ is a non-singular), it becomes
\begin{align}
	{\mathbf{S}_W}^{-1}\mathbf{S_B} \mathbf{W}=\mathbf{W}\Lambda.
\end{align}
Then the Fisher’s criterion is maximized when the projection matrix $\mathbf{W}$ is composed of the eigenvectors of ${\mathbf{S}_W}^{-1}\mathbf{S_B}$, and $\Lambda$ are the associated eigenvalues. Notice that there will be at most $C-1$ eigenvectors with non-zero real corresponding eigenvalues. Each eigenvectors represents one axis of the LDA space, and the corresponding eigenvalues represents the discriminatory ability between different classes. Thus, the eigenvectors with the $k$ highest eigenvalues are used to construct a lower dimensional space.
\clearpage
% !TeX spellcheck = en_US
\section{CUR Decomposition\label{Sec:DR:CUR}}
CUR decomposition, developed by Mahoney and Drineas\cite{MahoneyPNAS2009}, finds a low-rank approximation of matrix $A$ as the product of three matrices $C$, $U$, and $R$, where $C$ is a matrix consisting of selected columns of the original matrix, $R$ is a matrix consisting of selected rows of the original matrix, and $U$ is a matrix that ideally reconstructs the original matrix from $C$ and $R$. Usually the CUR is designed to be a rank-$k$ approximation, which requires that $C$ contains $k$ columns of $A$, $R$ contains $k$ rows of $A$, and $U$ is a $k$-by-$k$ matrix. The CUR matrix decomposition technique was developed to provide more interpretable and computationally efficient alternatives to SVD in principal component analysis (PCA), despite the fact that CUR is usually less accurate than SVD.

The fundamental questions of the CUR decomposition methods are: 1) Which columns of $A$ should be used to build $C$? Which rows should be used for $R$? 2) How to obtain the best $U$ given $C$ and $R$? %Different strategies have been proposed, for instance column pivoted QR factorizations, volume optimization, uniform sampling of columns, leverage scores, and empirical interpolation approaches. 
\clearpage
% !TeX spellcheck = en_US
\section{Independent Component Analysis\label{Sec:DR:ICA}}
TODO: to check the equations.

Independent component analysis (ICA) is a special case of blind source separation, which is used for separating a multivariate signal into additive subcomponents. ICA is considered as an extension of the principal component analysis (PCA, see section~\ref{Sec:DR:PCA}) technique. However, PCA searches uncorrelated components while ICA looks for independent components. For a not very recent review, please refer to Ref.~\cite{HyvarinenNN2000}.

ICA is built based on three assumptions. 
\begin{enumerate}
	\item \textbf{Independence}: The source signals are independent of each other.
	\item \textbf{Non-Gaussianity}: The mixed signals are Gaussian, but the values in each source signal have non-Gaussian distributions.
	\item \textbf{Complexity}: Mixed signals are more complex than source signals. 
\end{enumerate}

The \textit{signals} must be preprocessed before they can be projected to find the unmixing matrix and the \textit{sources}. The preprocessing steps include centering, whitening and dimensionality reduction. Suppose a random $r$-dimensional vector $\mathbf{X}=(X_1,\dots,X_r)^T$ has been detected, of which the mean and the covariance matrix are $\operatorname{E}\{\mathbf{X}\}=\boldsymbol{\mu}$ and $\operatorname{cov}\{\mathbf{X}\}=\boldsymbol{\Sigma}_{\mathbf{XX}}$, respectively. These \textit{signals} should be preprocessed by first centering so that they have zero mean, and then by sphering (or whitening) so that the uncorrelated components have unit variances. Using spectral decomposition, we have $\boldsymbol{\Sigma}_{\mathbf{XX}}=\mathbf{U}\boldsymbol{\Lambda}\mathbf{U}^{\operatorname{T}}$, where the columns of the unitary matrix $\mathbf{U}$ are the eigenvectors of $\boldsymbol{\Sigma}_{\mathbf{XX}}$, and the diagonal elements of the diagonal matrix $\boldsymbol{\Lambda}$ are the corresponding eigenvalues. The centered and sphered version of $\mathbf{X}$ can be given by
\begin{equation}
    \mathbf{Z}\leftarrow \boldsymbol{\Lambda}^{-1/2}\mathbf{U}^{\operatorname{T}}(\mathbf{X}-\boldsymbol{\mu}).
\end{equation}
In the above, we have assumed that both $\boldsymbol{\mu}$ and $\boldsymbol{\Sigma}_{\mathbf{XX}}$ are known. When they are unknown as in most cases in practice, we take $n$ observations, $\mathbf{x}_1, \dots, \mathbf{x}_n$, on $\mathbf{X}$ to compute $\bar{\mathbf{x}}=n^{-1}\sum_{i=1}^n\mathbf{x}_i$ and $\hat{\boldsymbol{\Sigma}}_{\mathbf{XX}}=n^{-1}\sum_{i=1}^n(\mathbf{x}_i-\bar{\mathbf{x}})(\mathbf{x}_i-\bar{\mathbf{x}})^{\operatorname{T}}=\hat{\mathbf{U}}\hat{\boldsymbol{\Lambda}}\hat{\mathbf{U}}^{\operatorname{T}}$. To reduce the dimensionality of the data, only the first $J < r$ sphered variables (corresponding to the eigenvectors with the largest magnitudes of the eigenvalues) are retained, where $J$ is chosen to explain a certain (high) proportion of the total variance as we do in PCA.

The observed data set $\mathbf{X}=(X_1,\dots,X_r)^{\operatorname{T}}$ are generated by
\begin{equation}
	\mathbf{X}=f(\mathbf{S})+\mathbf{e},
	\label{eq:DR:ICA:signalfromsource}
\end{equation}
where $\mathbf{S}=(S_1,\dots,S_m)^{\operatorname{T}}$ is an unknown vector of source whose components are independent latent variables, $f:\mathbb{R}^m\to \mathbb{R}^r$ is an unknowing mixing function, and $\mathbf{e}$ represents measurement noise with a zero mean. We assume that $\boldsymbol{\operatorname{E}}(\mathbf{S})=\mathbf{0}$ and $\boldsymbol{\operatorname{cov}}(\mathbf{S})=\mathbf{I}_m$. 

\textit{Nomenclature}: If $f$ is taken to be a linear (nonlinear) function, Eq.~\ref{eq:DR:ICA:signalfromsource} is described as a \textit{linear} (\textit{nonlinear}) ICA model. In most applications of ICA, it is assumed that the additive noise $\mathbf{e}$ is zero, and all noise in the model is to be associated with the components of the source. Such a model is referred to as \textit{noiseless} ICA. Otherwise, it is referred to as \textit{noisy} ICA. In most ICA applications, $\mathbf{X}$ is regarded as a stochastic process $\mathbf{X}(t) = (X_1(t), \dots, X_r(t))^{\operatorname{T}}$, where $t$ is a time or index parameter. In the linear noiseless ICA model with temporally structured sources and \textit{static} (time-independent) mixing, the model is written as $\mathbf{X}(t)=\mathbf{A}\mathbf{S}(t)$, where $\mathbf{S}(t)$ is assumed to be a \textit{stationary sources}. If $\mathbf{A}$ is also time dependent, this model is referred to as \textit{dynamic mixing}. In the following, we only consider \textit{static} mixing.

With the observed data set $\mathbf{X}=(X_1,\dots,X_r)^{\operatorname{T}}$ as an input, the task of ICA is to transform $\mathbf{X}$ into a vector of source with maximally independent components $\mathbf{Y}=(Y_1,\dots,Y_m)^{\operatorname{T}}$ using a linear static transformation $\mathbf{W}$ as $\mathbf{Y}=\mathbf{WX}$. The independence is measured by some function $F(\mathbf{Y})$. ICA finds the independent components (also called factors, latent variables or sources) by maximizing the statistical independence of the estimated components. There are many ways to define a proxy for independence, and the choice governs the form of the ICA algorithm, such as by 1) minimization of mutual information, which used measures like Kullback-Leibler Divergence and maximum entropy, 2) maximization of non-Gaussianity, which uses kurtosis and negentropy, and 3) using maximum likelihood estimation method. 

\subsection{Maximizing the non-Gaussianity\label{Sec:DR:ICA:MnG}}
Non-Gaussianity can be measured by kurtosis and negative entropy.
\subsubsection{Kurtosis\label{Sec:DR:ICA:MnG:Kurtosis}}
The extracted signal (sources) can be extracted by finding the orientation of the weight vectors which maximizes the kurtosis of the extracted signal. \textit{Although it is simple to calculate, it is sensitive to outliers. Therefore, it is not a robust way to measure the non-Gaussianity.} The kurtosis ($K$) for any probability density function is defined as
\begin{equation}
	K(\mathbf{X})=\operatorname{E}\left[\mathbf{X}^4\right]-3\left[\operatorname{E}\left[\mathbf{X}^2\right]\right]^2.
\end{equation}
The normalized kurtosis ($\hat{K}$) is the ratio between the fourth and second central moments, and it is given by
\begin{equation}
	\hat{K}=\frac{\operatorname{E}\left[\mathbf{X}^4\right]}{\left[\operatorname{E}\left[\mathbf{X}^2\right]\right]^2}-3\approx \frac{\frac{1}{N}\sum_{i=1}^N\left(X_i-\mu\right)^4}{\left(\frac{1}{N}\sum_{i=1}^N\left(X_i-\mu\right)^2\right)^2}-3
\end{equation}
For whitened data $\mathbf{Z}$ with a unit variance and zero mean,
\begin{equation}
	K(\mathbf{Z})=\hat{K}(\mathbf{Z})=\operatorname{E}\left[\mathbf{Z}^4\right]-3.
\end{equation}

The ICs can be found by maximizing kurtosis of extracted signals $\mathbf{Y}=\mathbf{W}^{\operatorname{T}}\mathbf{Z}$, which can be written as
\begin{equation}
	K(\mathbf{Y})={\operatorname{E}}\left[\left(\mathbf{W}^{\operatorname{T}}\mathbf{Z}\right)^4\right]
\end{equation}
with the gradient being
\begin{equation}
	\frac{\partial K(\mathbf{W}^{\operatorname{T}}\mathbf{Z})}{\partial \mathbf{W}}=c{\operatorname{E}}\left[\mathbf{Z}\left(\mathbf{W}^{\operatorname{T}}\mathbf{Z}\right)^3\right].
\end{equation}
The weight vector is updated iteratively via
\begin{equation}
	\mathbf{w}_{new}=\mathbf{w}_{old}+\eta {\operatorname{E}}\left[\mathbf{Z}\left(\mathbf{W}^{\operatorname{T}}\mathbf{Z}\right)^3\right],
\end{equation}
and
\begin{equation}
	\mathbf{w}_{new}=\frac{\mathbf{w}_{new}}{\lVert\mathbf{w}_{new}\rVert},
\end{equation}
since $\lVert\mathbf{w}\rVert=1$.

\subsubsection{Negative entropy\label{Sec:DR:ICA:MnG:ne}}
Negative entropy, or negentropy for short, is defined as
\begin{equation}
	J(\mathbf{Y})=H(\mathbf{Y}_{Gaussian})-H(\mathbf{Y}),
\end{equation}
where $H(\mathbf{Y}_{Gaussian})$ is the entropy of a Gaussian random variable whose covariance matrix is equal to the covariance matrix of $\mathbf{Y}$. The entropy of a random variable $\mathbf{Y}$ which has $N$ possible outcomes is
\begin{equation}
	H(\mathbf{Y})=-\operatorname{E}\left[\log{p_y(y)}\right]=-\sum_i^N p_y(y^i)\log{p_y(y^i)},
\end{equation}
where $p_y(y^i)$ is the probability of the event $y^i,\, i=1,\dots, N$. The negentropy is always nonnegative because the entropy of a Gaussian variable is the maximum among all other random variables with the same variance. It is zero only when all variables are Gaussian, i.e., $H(y_{Gaussian})=H(y)$. Moreover, it is invariant for invertible linear transformation and scale-invariant. However, calculating the entropy from a finite data is computationally difficult. Hence, different approximations have been introduced. For example,
\begin{equation}
	J(y)\approx \sum_{i=1}^p k_i\left({\operatorname{E}}\left[G_i(y)\right]-{\operatorname{E}}\left[G_i(v)\right]\right)^2,
\end{equation} 
where $k_i$ are some positive constants, $v$ indicates a Gaussian variable with zero mean and unit variance, $G_i$ represent some quadratic function. The function $G$ has different choices such as
\begin{equation}
	G_1(y)=\frac{1}{a}\log\cosh (a_1y) \quad G_2(y)=-\exp(-y^2/2),
\end{equation}
where $1\leq a_1 \leq 2$.

\subsection{Minimization of mutual information\label{Sec:DR:ICA:MMI}}
The amount of mutual information between the $m$ components of $\mathbf{Y}$ can be written as
\begin{equation}
	I(\mathbf{Y})=c-\sum_{j=1}^m J(Y_j),
\end{equation}
where $c=mH(\mathbf{Y}_{Gaussian})-H(\mathbf{X})$ does not depend on the unmixing matrix $\mathbf{W}$ and is a constant. Therefore, minimizing the mutual information between the components of $\mathbf{Y}$ is equivalent to maximizing the sum of the negentropies of the independent components of $\mathbf{Y}$.

\subsection{Maximum likelihood\label{Sec:DR:ICA:ML}}
For a noise-free ICA model, $\mathbf{X}=\mathbf{AS}$. Hence,
\begin{equation}
	P_\mathbf{X}(\mathbf{X})=\frac{p_\mathbf{S}(\mathbf{S})}{|{\operatorname{det}}\mathbf{A}|}=|{\operatorname{det}}\mathbf{W}|p_\mathbf{S}(\mathbf{S}).
\end{equation}
For independent source signals, $p_\mathbf{S}(\mathbf{S}=\prod_ip_i(\mathbf{s}_i)$, $P_\mathbf{X}(\mathbf{X})$ becomes
\begin{equation}
	P_\mathbf{X}(\mathbf{X})=|{\operatorname{det}}\mathbf{W}|\prod_ip_i(\mathbf{s}_i)=|{\operatorname{det}}\mathbf{W}|\prod_ip_i(\mathbf{w}_i^T\mathbf{X}).
\end{equation}
Given $T$ observations of $\mathbf{X}$, the likelihood of $\mathbf{W}$ is given by
\begin{equation}
	\mathcal{L}(\mathbf{W})=\prod_t^T\prod_i^p|{\operatorname{det}}\mathbf{W}|p_i(\mathbf{w}_i^T\mathbf{x}(t)).
\end{equation}
Usually, a log-likelihood is preferred:
\begin{equation}
	{\operatorname{log}}\mathcal{L}(\mathbf{W})=\sum_t^T\sum_i^p{\operatorname{log}}p_i(\mathbf{w}_i^T\mathbf{x}(t))+T\operatorname{log}|{\operatorname{det}}\mathbf{W}|,
\end{equation}
or
\begin{align}
	\frac{1}{T}{\operatorname{log}}\mathcal{L}(\mathbf{W})=&{\operatorname{E}}\left[\sum_t^T\sum_i^p{\operatorname{log}}p_i(\mathbf{w}_i^T\mathbf{x}(t))\right]+\operatorname{log}|{\operatorname{det}}\mathbf{W}|,\notag\\
	=&-\sum_i H(\mathbf{w}_i^T\mathbf{X})+\operatorname{log}|{\operatorname{det}}\mathbf{W}|.
\end{align}
Therefore, the likelihood and mutual information are approximately equal, and they only differ by the sign and an additive constant.
\clearpage
% !TeX spellcheck = en_US
\section{Isometric Feature Mapping (Isomap)\label{Sec:DR:Isomap}}
Isomap, introduced in 2000 for the first time, is a nonlinear generalization of the MDS algorithm in which Euclidean distances are replaced by geodesic distances.\cite{TenenhaumScience2000} Isomap seeks a mapping such that the geodesic distance between data points match the corresponding Euclidean distance in the transformed space. However, the geometric structure of the given data is unknown usually. In order to obtain the geodesic distance between the points, it has been assumed that, in a small neighborhood, the Euclidean distance is a good approximation for the geodesic distance. While for the points far apart, the geodesic distance is approximated as the sum of Euclidean distances along the shortest connecting path.

The first step is to build a weighted neighborhood graph $G(\mathcal{V},\mathcal{E})$ from the given data by connecting only nearby points, where the vertices or nodes, $\mathcal{V}=\{\mathbf{x}_1,\dots,\mathbf{x}_n\}$, are the input data and the edges, $\mathcal{E}=\{e_{ij}\}$, indicate the neighborhood relationship between the points. The weight $w_{ij}$ of edge $e_{ij}$ equals to the distance $d_{ij}$ between those points if they are close to each, or $0$ otherwise. Closeness is defined either by the $\epsilon$-approach, if $\lVert\mathbf{x}_i-\mathbf{x}_j\rVert<\epsilon$ where $\epsilon >0$, or by the $K$-nearest neighbors. Then, Dijkstra’s algorithm or Floyd's algorithm is applied with the nearest neighbor graph $G$ to find the shortest-path distances ($d_G(i,j)$) for all pairs of data points. Finally, MDS is applied to the distance matrix $d_G(i,j)$ to find a $k$-dimensional representation $\mathbf{Y}$ of the original data.

Isomap algorithm is sensitive to noise in the data.
\clearpage
% !TeX spellcheck = en_US
\section{Locally Linear Embedding (LLE)\label{Sec:DR:LLE}}
Locally linear embedding was introduced by Roweis and Saul in 2000.\cite{RoweisScience2000} It differs from Isomap by eliminating the need to estimate pairwise distances between widely separated data points.

Suppose sufficient data have been sampled from some underlying manifold, which are denoted as $\{\mathbf{x}_i\}\in \mathbb{R}^D, \text{for }i\in [1,N]$. It can be expected that each data point and its neighbors lie on or close to a locally linear patch of the manifold. Then each data point is reconstructed linearly from its neighbors, and the reconstruction errors measured by the cost function
\begin{equation}
	\epsilon(\mathbf{W})=\sum_i\lVert \mathbf{x}_i-\sum_j W_{ij}\mathbf{x}_j\rVert^2.
\end{equation}
The weights $\mathbf{W}$ can be obtained by minimizing the cost function subject to two constraints: first, $W_{ij}=0$ if $\mathbf{x}_j$ does not belong to the set of neighbors of $\mathbf{x}_i$; second, the rows of $\mathbf{W}$ sum to one: $\sum_j W_{ij}=1$. Formally, we minimize the Lagrangian function
\begin{equation}
	f(\mathbf{x}_i)=\mathbf{w}_i^{\operatorname{T}}G_i\mathbf{w}_i-\lambda \left(\mathbf{1}_n^{\operatorname{T}}\mathbf{w}_i-1\right)
\end{equation}
with respect to $\mathbf{w}_i$, where $G_i=(G_{i,jk})$ is an $(n\times n)$ Gram matrix with $G_{i,jk}=(\mathbf{x}_i-\mathbf{x}_j)^{\operatorname{T}}(\mathbf{x}_i-\mathbf{x}_k)$. Minimizing $f(\mathbf{x}_i)$ with respect to $\mathbf{w}_i$ yields
\begin{equation}
	\widehat{\mathbf{w}}_i=\frac{\lambda}{2}G_{i}^{-1}\mathbf{1}_n.
\end{equation}
Multiplying both sides of this equation from the left by $\mathbf{1}_n^{\operatorname{T}}$ and using the normalization condition $\mathbf{1}_n^{\operatorname{T}}\mathbf{w}_i=1$, we find
\begin{equation}
	\frac{\lambda}{2}=\frac{1}{\mathbf{1}_n^{\operatorname{T}}\mathbf{G}_i^{-1}\mathbf{1}_n}.
\end{equation}
Then, the optimal weights can be rewritten as
\begin{equation}
	\widehat{\mathbf{w}}_i=\frac{\mathbf{G}_i^{-1}\mathbf{1}_n}{\mathbf{1}_n^{\operatorname{T}}\mathbf{G}_i^{-1}\mathbf{1}_n}.
\end{equation}

In the final step of this algorithm, each high-dimensional data points $\mathbf{x}_i$ is mapped to a low-dimensional observation $\mathbf{y}_i$ by minimizing the embedding cost function
\begin{equation}
	\phi(\mathbf{Y})=\sum_i \lVert \mathbf{y}_i -\sum_j W_{ij}\mathbf{y}_j\rVert ^2
\end{equation}
with fixed $\mathbf{W}$. It can be converted into a $N\times N$ eigenvalue problem by the transformation
\begin{equation}
	\phi(\mathbf{Y})=\sum_{ij}(\delta_{ij}-W_{ij}-W_{ji}+\sum_k W_{ki}W_{kj})\mathbf{y}_i\cdot \mathbf{y}_j,
\end{equation}
in which we have assumed/forced
\begin{equation}
	\sum_i \mathbf{y}_i=0\quad \text{and} \quad \frac{1}{N}\sum_i \mathbf{y}_i\otimes \mathbf{y}_i=\mathbf{I}.
\end{equation}
The smallest eigenvalue is zero with corresponding eigenvector $\mathbf{v}_n=n^{-1/2}\mathbf{1}_n$. The next $d$ smallest eigenvectors define the embedding coordinates.
\clearpage
% !TeX spellcheck = en_US
\section{Laplacian Eigenmaps\label{Sec:DR:LE}}
Laplacian eigenmaps was developed by Belkin and Niyogi in 2001.\cite{BelkinANIPS2001}

Given $k$ points $\mathbf{x}_1, \mathbf{x}_2,\dots, \mathbf{x}_k$ in $\mathbb{R}^t$, a weighted graph $G=(V,E)$ with $k$ nodes, one for each point, and a set of edges connecting neighboring points to each other are constructed. Whether two points are neighbors of each other is determined by their closeness in either of the two ways:
\begin{itemize}
	\item $\epsilon$-neighborhoods: With a prechosen parameter $\epsilon$, nodes $i$ and $j$ are connected by an edge if $\lVert \mathbf{x}_i-\mathbf{x}_j\rVert^2<\epsilon$.
	\item $n$-nearest neighbors: Given parameter $n$, Nodes $i$ and $j$ are connected by an edge if $i$ is among $n$ nearest neighbors of $j$ or $j$ is among $n$ nearest neighbors of $i$.
\end{itemize}

Now the weight for each edge can be calculated in either of the two ways:
\begin{itemize}
	\item Heat kernel: If nodes $i$ and $j$ are connected, set
	\begin{equation}
	    W_{ij}=e^{-\frac{\lVert \mathbf{x}_i-\mathbf{x}_j\rVert^2}{t}}
	\end{equation}
	
	\item Simple-minded: $W_{ij}=1$ if and only if vertices $i$ and $j$ are connected by an edge.
\end{itemize}

Compute eigenvalues and eigenvectors for the generalized eigenvector problem:
\begin{equation}
    L\mathbf{y}=\lambda D\mathbf{y}
\end{equation}
where $D$ is a diagonal weight matrix with $D_{ii}=\sum_j W_{ji}$, $L=D-W$ is the Laplacian matrix. let $\mathbf{y}_0, \dots, \mathbf{y}_{k-1}$ be the solutions of the equation above, ordered by their eigenvalues with $\mathbf{y}_0$ having the smallest eigenvalue (in fact 0). The image of $\mathbf{x}_i$ under the embedding into the lower dimensional space $\mathbf{R}^m$ is given by $(\mathbf{y}_1(i),\dots, \mathbf{y}_m(i))$. 

Justification for the process above is that the points connected on the graph should stay as close as possible after embedding, which means we should minimize the objective function
\begin{equation}
	\sum_{i,j\in E}(\mathbf{y_i}-\mathbf{y}_j)^2W_{ij}
\end{equation}
with respect to $\mathbf{y}_1, \dots, \mathbf{y}_n$ subject to appropriate constraints. It means that when $W_{ij}$ is large, $\mathbf{x}_i$ and $\mathbf{x}_j$ are very close to each other. Then, $\mathbf{y}_i$ and $\mathbf{y}_j$ must still be close. On the contrary, when $W_{ij}$ is small, meaning that $\mathbf{x}_i$ and $\mathbf{x}_j$ are far away from each other, then there is much flexibility in putting $\mathbf{y}_i$ and $\mathbf{y}_j$ on the line. For any $\mathbf{y}$, we have
\begin{align}
	\frac{1}{2}\sum_{i,j\in E}(\mathbf{y}_i-\mathbf{y}_j)^2 W_{ij}=&\frac{1}{2}\sum_{i,j\in E}(\mathbf{y}_i^2+\mathbf{y}_j^2-2\mathbf{y}_i\mathbf{y}_j)W_{ij}\notag\\
	=&\frac{1}{2}\sum_{i}\mathbf{y}_i^2 D_{ii}+\frac{1}{2}\sum_j \mathbf{y}_j^2 D_{jj}-\sum_{i,j\in E}\mathbf{y}_i\mathbf{y}_j W_{ij}\notag\\
	=&\sum_i \mathbf{y}_i^2D_{ii} -\sum_{i,j\in E}\mathbf{y}_i\mathbf{y}_jW_{ij}\notag\\
	=&\sum_{i,j\in E}\mathbf{y}_iD_{i,j}\mathbf{y}_j-\sum_{i,j\in E}\mathbf{y}_i\mathbf{y}_jW_{ij}\notag\\
	=&\sum_{i,j\in E}\mathbf{y}_i(D_{ij}-W_{ij})\mathbf{y}_j\notag\\
	=&\mathbf{y}^{\operatorname{T}}L\mathbf{y}.
\end{align}

Therefore, the minimization problem reduces to $\operatorname*{arg\,min}\limits_{\substack{\mathbf{y}\\\mathbf{y}^{\operatorname{T}}D\mathbf{y}=1}} \mathbf{y}^{\operatorname{T}}L\mathbf{y}$. The constraint $\mathbf{y}^{\operatorname{T}}D\mathbf{y}=1$ removes an arbitrary scaling factor in the embedding. $L$ is semi-definite defined, and the vector $\mathbf{y}$ that minimizes the objective function is given by the minimum eigenvalue solution to the generalized eigenvalue problem $L\mathbf{y}=\lambda D\mathbf{y}$. Alternatively, using the Lagrangian multiplier and minimization with respect to $\mathbf{y}$, we have
\begin{equation}
	\frac{\partial}{\partial \mathbf{y}}\left[\mathbf{y}^{\operatorname{T}}L\mathbf{y}+\lambda\left(\mathbf{y}^{\operatorname{T}}D\mathbf{y}-1\right)\right]=0,
\end{equation}
and it leads to
\begin{equation}
	L\mathbf{y}=\lambda D\mathbf{y}.
\end{equation}

Since All the rows (and columns) sum to 0, it is easy to see that $\mathbf{y}=\mathbf{1}$ (all 1s) si an eigenvector with eigenvalue 0. To eliminate this trivial solution which collapses all vertices of $G$ onto the real number 1, an additional constraint of orthogonality must be imposed to obtain
\begin{equation}
	\mathbf{y}_{opt}=\operatorname*{arg\,min}\limits_{\substack{\mathbf{y}\\\mathbf{y}^{\operatorname{T}}D\mathbf{y}=1\\\mathbf{y}^{\operatorname{T}}D\mathbf{1}=0}} \mathbf{y}^{\operatorname{T}}L\mathbf{y}.
\end{equation}

Thus, the solution $\mathbf{y}_{opt}$ is now given by the eigenvector with smallest non-zero eigenvalue. More generally, the embedding of the graph into $\mathbb{R}^m $ ($m>1$) is given by the $n\times m$ matrix $Y=\left[\mathbf{y}_1\,\mathbf{y}_2\cdots \mathbf{y}_m\right]$. It reduces to
\begin{equation}
	Y_{opt}=\operatorname*{arg\,min}\limits_{\substack{Y\\Y^{\operatorname{T}}DY=I}} \operatorname{tr}{\left(Y^{\operatorname{T}}LY\right)}.
\end{equation} 

\clearpage
% !TeX spellcheck = en_US
\section{Diffusion Maps\label{Sec:DR:DM}}
Diffusion maps was developed by Coifman et al. in 2005\cite{CoifmanPNAS2005a,CoifmanPNAS2005b,CoifmanACHA2006}, which computes a family of embeddings of a data set into Euclidean space (often low-dimensional) whose coordinates can be computed from the eigenvectors and eigenvalues of a diffusion operator on the data, while ensuring that the diffusion distance in the original space between points is well approximated by the Euclidean distance in the reduced-dimensional space. Diffusion maps are part of the family of nonlinear dimensionality reduction methods which focus on discovering the underlying manifold that the data has been sampled from. By integrating local similarities at different scales, diffusion maps give a global description of the data-set. Compared with other methods, the diffusion maps algorithm is robust to noise perturbation and computationally inexpensive.

Given a set of $N$ data points $\mathbf{x}=\{x_1, x_2,\dots,x_N\}$, the connectivity between data points $x_i$ and $x_j$ is defined by the transition probability between these two points $p(x_i,x_j)$, which is measured by their distance
%\begin{equation}
%	p(x_i,x_j)\propto \begin{cases}
%		k(x_i,x_j), & \text{if } k(x_i,x_j) >\epsilon\\
%		0, & \text{otherwise}
%	\end{cases},
%\end{equation}
\begin{equation}
	p(x_i,x_j)\propto k(x_i,x_j).
\end{equation}
%with $0<\epsilon \ll 1$. 
The kernel $k(x,y)$ defines a local measure of similarity within a certain neighborhood. Since a given kernel will capture a specific feature of the data set, it should be well-designed to match the requirement of the application. A Gaussian kernel
\begin{equation}
	k(x,y)=\exp{\left(-\frac{\lVert x-y\rVert^2}{\alpha}\right)}
\end{equation}
is frequently used, where $\alpha$ tunes the size of the neighborhood. Transition probability matrix $\mathbf{P}$ must be row-normalized, which leads to
%\begin{equation}
%	p(x_i,x_j)= \begin{cases}
%		k(x_i,x_j)/\sum\limits_{x_j\in S_i}k(x_i,x_j), & \text{if } k(x_i,x_j) >\epsilon\\
%		0, & \text{otherwise}
%	\end{cases}.
%\end{equation}
%$S_i$ is the neighborhood of $x_i$.
\begin{equation}
	p(x_i,x_j)= k(x_i,x_j)/\sum\limits_{x_j\in \mathbf{x}}k(x_i,x_j),
\end{equation}
or in matrix form
\begin{equation}
	P=D^{-1}K,
\end{equation}
where $D$ is the diagonal matrix consisting of the row-sums of $K$. \textit{Note}: $P$ is non-symmetric, which has a leading eigenvalue $\lambda_1=1$ with multiplicity 1.
 
Suppose there are three data points $\{x_1, x_2, x_3\}$ and the single-step transition probability matrix is
\begin{equation}
  P=\begin{bmatrix}
	p_{11} & p_{12} & 0\\
	p_{21} & p_{22} & p_{23} \\
	0      & p_{32} & p_{33}
\end{bmatrix},
\end{equation}
where we have assumed that the transition probability from $x_1$ to $x_3$ is zero and vice versa. It can be easily found that
\begin{equation}
  P^2=
  \begin{bmatrix}
  	p_{11}p_{11}+p_{12}p_{21} & p_{11}p_{12}+p_{12}p_{22} & p_{12}p_{23}\\
  	p_{21}p_{11}+p_{22}p_{21} & p_{21}p_{12}+p_{22}p_{22}+p_{23}p_{32} & p_{22}p_{23}+p_{23}p_{33} \\
  	p_{32}p_{21}      & p_{32}p_{22}+p_{33}p_{32} & p_{32}p_{23}+p_{33}p_{33}
  \end{bmatrix}.
\end{equation}
It shows that after two steps of propagation, the probability of transition from $x_1$ to $x_3$ is no longer zero, instead it equals to the probability of transition from $x_1$ to $x_2$, $p_{12}$, in the first step multiplied by the probability of transition from $x_2$ to $x_3$, $p_{23}$, in the second step. Similar observation can also be found for other transitions among the data points.

It can be seen from the above that $P^t_{ij}$ sum all paths of length $t$ from point $x_i$ to point $x_j$. With increasing value of $t$, different scales of the data structure can be visualized. This is the diffusion process, during which the local connectivity is integrated to present the global connectivity of a data set. Besides, pathways built by long, low probability jumps are gradually replaced by short, high probability jumps. 

With the global geometric structure of a data set uncovered by the diffusion process described above, a diffusion metric can be defined for this structure. The metric measures the similarity of two points in the observed space as the connectivity between them, and is defined as
\begin{align}
	D_t(x_i,x_j)^2=&\sum_{u\in X}|p_t(x_i,u)-p_t(x_j,u)|^2\notag\\
	              =&\sum_k |P_{ik}^t-P_{kj}^t|^2.
\end{align}
$D_t(x_i,x_j)^2$ is known as the diffusion distance between point $x_i$ and $x_j$. In order to have a small diffusion distance between two points, there should be many high probability paths of length $t$ linking these two points. Since it sums over all possible paths, this algorithm is robust to noise perturbation. Besides, the path probabilities between $x_i$, $u$ and $u$, $x_j$ must be roughly equal. This happens when both $x_i$ and $x_j$ are well connected via $u$. However, calculating diffusion distances is computational expensive. Therefore, it is convenient to map data points into a Euclidean space, in which the diffusion distance becomes the Euclidean distance in this new diffusion space. 

Diffusion maps maps coordinates between data and diffusion space by reorganizing data according to the diffusion metric. It preserves a data set's intrinsic geometry with data points mapped to a lower-dimensional structure. With the mapping 
\begin{equation}
	y_i := \begin{bmatrix}
		p_t(x_i, x_{1}) \\
		p_t(x_i, x_{2}) \\
		\vdots \\
		p_t(x_i, x_{N})
	\end{bmatrix}=P_{i\ast}^{\operatorname{T}},
\end{equation}
the Euclidean distance between two mapped points, $y_i$ and $y_j$, is
\begin{align}
	\lVert y_i-y_j\rVert_E^2=&\sum_{u\in X} |p_t(x_i,u)-p_t(x_j,u)|^2\notag\\
	                        =&\sum_k |P_{ik}^t-P_{kj}^t|^2\notag\\
	                        =&D_t(x_i,x_j)^2.
\end{align}
However, this is not a good mapping for dimension reduction, and we must transform this mapping into a new coordinate system. As said above, $P$ is non-symmetric. Let its left and right eigenvectors be $\left\{e_k\right\}$ and $\left\{v_k\right\}$ and the eigenvalues $\left\{\lambda_k\right\}$. The eigen decomposition
\begin{equation}
	P=\sum_k\lambda_kv_ke_k^{\operatorname{T}}
\end{equation}
indicates that each row of the diffusion matrix $P$ can be expressed in terms of a new basis $\left\{e_k\right\}$, the left eigenvectors of $P$. In this new coordinate system, each row of $P$ is represented by a point
\begin{equation}
	y_i^\prime=\begin{bmatrix}
		\lambda_1^t\phi_1(i) \\
		\lambda_2^t\phi_2(i) \\
		\vdots \\
		\lambda_n^t\phi_n(i)
	\end{bmatrix},
\end{equation}
where $\phi_k(s)$ indicates the $s$th element of the $k$th eigenvector of $P$. The Euclidean distance between mapped points $y_i^\prime$ and $y_j^\prime$ is the diffusion distance. In most cases, $\lambda_k$ decays very fast. Therefore, dimension reduction can be achieved by retaining the $m$ dimensions associated with the dominant eigenvectors.
\clearpage
% !TeX spellcheck = en_US
\section{t-Distributed Stochastic Neighbor Embedding Algorithm (t-SNE)\label{Sec:DR:t-SNE}}
The t-distributed Stochastic Neighbor Embedding (t-SNE) algorithm, proposed by Laurens van der Maaten and Geoffrey Hinton in the year 2008,\cite{vandermaatenJMLR2008} was an improved version of the SNE algorithm developed by Hinton and Roweis in 2002.\cite{HintonNIPS2002}

In SNE, the high-dimensional Euclidean distances between data points are converted by conditional probabilities that represent similarities. The similarity of data point $x_j$ to data point $x_i$ is the conditional probability, $p_{j|i}$, that $x_i$ would take $x_j$ as its neighbor, which is proportional to their probability density under a Gaussian centered at $x_i$
\begin{equation}
	p_{j|i}=\frac{\exp{\left(-\lVert x_i-x_j\rVert^2/2\sigma_i^2\right)}}{\sum_{k\neq i}\exp{\left(-\lVert x_i-x_j\rVert^2/2\sigma_i^2\right)}}.
\end{equation}
The magnitudes of $\{\sigma_i\}$ tune the structure of the connections.
Similarly, for the low-dimensional mapped data points $y_i$ and $y_j$, the conditional probability, denoted as $q_{j|i}$, is written also in a Gaussian form
\begin{equation}
	q_{j|i}=\frac{\exp{\left(-\lVert y_i-y_j\rVert^2\right)}}{\sum_{k\neq i}\exp{\left(-\lVert y_i-y_j\rVert^2\right)}}
\end{equation}
with variance set to $1/\sqrt{2}$.

If the mapped data points $y_i$ and $y_j$ faithfully model the similarity between the data points $x_i$ and $x_j$, the conditional probabilities $p_{j|i}$ and $q_{j|i}$ will be equal. A natural way to measure the faithfulness is the Kullback--Leibler divergence. SNE minimizes the sum of Kullback--Leibler divergence over all data points
\begin{equation}
	C=\sum_i KL(P_i\Vert Q_i)=\sum_i \sum_j p_{j|i} \log{\frac{p_{j|i}}{q_{j|i}}}
\end{equation}
using a gradient descent method. Because the Kullback--Leibler is not symmetric, the cost function of SNE focuses on retaining the local structure of the data point in the map.

The minimization of the cost function $C$ is performed using a gradient descent method with the gradient
\begin{equation}
	\frac{\delta C}{\delta y_i}=2\sum_j(p_{j|i}-q_{j|i}+p_{i|j}-q_{i|j})(y_i-y_j).
\end{equation}
To escape from poor local minima, a relatively large momentum term is added to the gradient, and the update of $\mathcal{Y}$ is written as
\begin{equation}
  \mathcal{Y}^{(t)}=\mathcal{Y}^{(t-1)}+\eta \frac{\delta C}{\delta \mathcal{Y}}+\alpha(t)(\mathcal{Y}^{(t-1)}-\mathcal{Y}^{(t-2)}).
\end{equation}  
Despite this optimization strategy, the method still does not ensure that the global best is obtained. Therefore, it is common to run the optimization several times on the data set with different initial condition and random numbers.

Usually, it is unlikely that a single value of $\sigma_i$ will be optimal for all the data points. Instead, it is related to the distribution density of the data points, that is varying in the high-dimensional space. In SNE, the value of $\sigma_i$ is determined by a fixed perplexity specified by the user. The perplexity can be interpreted as a smooth measure of the effective number of neighbors, which is defined as
\begin{equation}
	perp(P_i)=2^{H(P_i)},
\end{equation} 
where $H(P_i)$ is the Shannon entropy of $P_i$ measured in bits
\begin{equation}
	H(P_i)=-\sum_j p_{j|i}\log_2{p_{j|i}}.
\end{equation}

Because of the asymmetry of the Kullback--Leibler divergence in the cost function, SNE often suffers from the ``crowding problem''. In order to solve this problem, t-SNE algorithm utilizes symmetrized version of the cost function and a heavily-tailed Student-t distribution rather than a Gaussian to compute the similarity between two points \textit{in the low-dimensional space}. 

As an alternative to minimizing the sum of the Kullback--Leibler divergences between the conditional probabilities $p_{j|i}$ and $q_{j|i}$, it is also possible to minimizes a single Kullback-Leibler divergence between a joint probability distribution, $P$, in the high-dimensional space and a joint probability distribution, $Q$, in the low-dimensional space:
\begin{equation}
	C=KL(P\Vert Q)=\sum_i\sum_j p_{ij}\log \frac{p_{ij}}{q_{ij}},
\end{equation}
in which
\begin{equation}
	p_{ij}=\frac{\exp{\left(-\lVert x_i-x_j\rVert^2/2\sigma^2\right)}}{\sum_{k\neq l}\exp{\left(-\lVert x_k-x_l\rVert^2/2\sigma^2\right)}}
\end{equation}
and
\begin{equation}
	q_{ij}=\frac{\exp{\left(-\lVert y_i-y_j\rVert^2\right)}}{\sum_{k\neq l}\exp{\left(-\lVert y_k-y_l\rVert^2\right)}}
	\label{eq:DR:t-SNE:symmetricQ}
\end{equation}
with $p_{ii}=q_{ii}=0$. However, t-SNE circumvent this problem by forcing the joint probabilities $p_{ij}$ in the high-dimensional space to be symmetric as $p_{ij}=\frac{p_{i|j}+p_{j|i}}{2n}$ and using Eq.~\ref{eq:DR:t-SNE:symmetricQ} for $q_{ij}$ in the low-dimensional space. The gradient of the cost function with respect to the mapped points becomes
\begin{equation}
	\frac{\delta C}{\delta y_i}=4\sum_j(p_{ij}-q_{ij})(y_i-y_j).
\end{equation}
\clearpage
% !TeX spellcheck = en_US
\section{Uniform Manifold Approximation and Projection (UMAP)\label{Sec:DR:UMAP}}
author killed by math

\clearpage
% !TeX spellcheck = en_US
\section{Spectral Gap optimization of Order Parameters\label{Sec:DR:SGOOP}}
Spectral Gap optimization of Order Parameters (SGOOP) was proposed by Tiwary and Berne in 2016.\cite{TiwaryPNAS2016}
\clearpage
% !TeX spellcheck = en_US
\section{Variational Approach for Conformation (VAC) Dynamics\label{Sec:DR:VAC}}

Variational approach for conformation (VAC) dynamics was developed by No\'e and N\"uske in 2013.\cite{NoeMMS2013}
